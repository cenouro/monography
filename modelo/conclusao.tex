\chapter{Conclusão}
\label{chap:conclusao}

\section{Benefícios para o crescimento profissional}

A oportunidade de trabalhar sem ser liderado por alguém da área de TI me interessou muito, pois vi isso como uma 
forma de adquirir experiência em diversas áreas, desde liderança (atualmente, a empresa contratou um outro estagiário para que me
ajude e seja meu subordinado), até coisas mais específicas da computação, como a parte de conversar com o cliente,
fazer o levantamento dos requisitos do sistema, e tomar decisões sobre linguagens de programação e metodologia de desenvolvimento.

Outro aspecto interessante e útil de estar trabalhando na empresa é a capacidade de aumentar meu \textit{networking}.
Estou envolvido diariamente com diversas pessoas, desde economistas, biólogos, médicos, publicitários, e até mesmo, como mencionado
anteriormente, com o ex-presidente do Banco do Brasil. Tal \textit{networking} diversificado me ajuda cada vez mais a aprimorar 
meu perfil de empreendedor e acredito que seja uma excelente forma de treinar a habilidade de definir produtos com futuros clientes
e trabalhar em áreas interdisciplinares.

%Comente como o estágio foi importante para o seu crescimento profissional. 

%Se houve problemas e/ou decepções, discuta-os nesta seção.


\section{Considerações sobre o curso de graduação}

O curso forneceu as bases teóricas para que o estágio fosse desenvolvido. Ainda que minha experiência no mercado de trabalho seja algo recente, 
as bases teóricas me facilitaram bastante a aquisição de novos conhecimentos com finalidade de elevar o meu trabalho desenvolvido 
durante o estágio ao nível profissional.

%Avalie o seu curso de graduação, em particular em termos de preparação para o mercado de trabalho. 

%Discuta também a importância das disciplinas do curso para o estágio realizado.

\section{Sugestões para o curso de graduação}

As disciplinas de engenharia de software que cursei não abordavam \gls{tdd}. Acredito que esta metodologia de desenvolvimento seja
muito importante e interessante, e deveria ser abordada.

Criptografia é um assunto pouco abordado no curso. Estudei criptografia apenas nas disciplinas de redes. Durante o estágio, foi necessário
utilizar criptografia em outras camadas da aplicação, como, por exemplo, no banco de dados. Acredito que criptografia seja um assunto
que deva ser abordado também nas disciplinas de bancos de dados.

As disciplinas de matemática não foram muito proveitosas. Não usei cálculo nem geometria analítica, por exemplo. 
Acredito que tais disciplinas ainda sejam muito importantes, pois fornecem bases lógicas e são requisitos para outras disciplinas. 
Mas a carga de matemática do curso, em minha opinião, deveria incluir mais disciplinas de matemática que possam ser usadas em computação 
diretamente do que disciplinas que são usadas em contextos mais específicos. Considero que cálculo seja uma disciplina de contexto 
específico, e disciplinas como matemática discreta e probabilidade sejam mais aplicáveis em áreas diversas da computação. Tive a 
oportunidade de cursar outras disciplinas do curso de matemática, fornecidas como optativas para a computação. Algumas delas 
foram Álgebra 1, Elementos de Matemática, Tópicos de Matemática Elementar, e Tópicos de Otimização Combinatória. O conteúdo de 
tais disciplinas é também fornecido no curso de computação, porém de forma muito mais resumida (nas disciplinas de matemática discreta). 
Acredito que as disciplinas de matemática discreta não deveriam ser tão condensadas/resumidas.

Outra disciplina que deveria receber maior atenção é a de empreendedores em informática. Considero que a linha divisória entre um 
funcionário e um empreendedor seja muito subjetiva. Eu, como funcionário, muitas vezes recorro a conceitos vistos na disciplina de 
empreendedorismo. Tal disciplina deveria ser obrigatória no curso.

As disciplinas preparatórias para a maratona de programação também são muito importantes. Pude ganhar mais experiência 
na parte de modelar o problema, testar, implementar e otimizar a implementação para que o programa seja executado de forma rápida. 
Nessas disciplinas, também melhorei minha habilidade de trabalhar em equipe, pois o time para a maratona é composto por três pessoas. 
Muitos dos conceitos absorvidos em tais disciplinas não são aplicados extensivamente, pois a prioridade no trabalho sendo desenvolvido é 
manutenibilidade do código, ao invés de código extremamente eficiente. Porém, tais disciplinas me deram uma melhor capacidade de 
resolver os problemas que podem surgir, e acredito que foram as responsáveis por interconectar as matérias teóricas com as matérias práticas.

A única reclamação que tenho sobre o curso é sobre a duração das aulas. A maioria das aulas tem duração de duas horas. 
Isso já é uma duração longa, e outras aulas podem ter até quatro horas de duração. É um contraste muito grande com as aulas do Massachusetts 
Institute of Technology, 
por exemplo, que tem duração de no máximo uma hora.

Em suma, estou satisfeito com o curso. Porém, tal satisfação só foi atingida porque busquei estudar assuntos extracurriculares. 
Infelizmente, não pude ver isso em muitos colegas, que se formaram antes de mim e que ainda assim sentiram que o curso não foi proveitoso. 
Acredito que seja possível reestruturar o curso de tal forma que futuros formandos entrem mais confiantes no mercado de trabalho.

%Apresente sugestões construtivas para o curso de graduação, justificando cada uma.

\section{Planos para o futuro}

Pretendo continuar trabalhando na empresa. Acredito que seja um lugar onde eu possa adquirir muita experiência,
principalmente em como trabalhar conjuntamente com profissionais de outras áreas.
%Comente quais são seus planos depois do término do estágio, a curto e médio prazos.
