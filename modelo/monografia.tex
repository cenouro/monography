\documentclass[12pt,times,a4paper,twoside]{icmc}

\usepackage[portuguese, brazilian]{babel}
\usepackage[utf8]{inputenc}
\usepackage[top=30mm,bottom=20mm,left=30mm,right=20mm,twoside]{geometry}
\usepackage[usenames,dvipsnames]{color}
\usepackage[nottoc]{tocbibind}
\usepackage{fancychap} % Remova se quiser tirar os detalhes do título do capítulo
\usepackage{indentfirst}
\usepackage{setspace}
\usepackage{graphicx}
\usepackage{epstopdf}
\usepackage{amssymb}
\usepackage{amsmath}
\usepackage{mathptmx} % Necessário para corrigir a fonte para Times
\usepackage{hyperref}
\usepackage{setspace}
\usepackage{algpseudocode}
\usepackage{pdfpages}
\usepackage{url}
\usepackage{listings}
\usepackage{natbib}
\lstset{numbers=left, %%% Para insercao de codigos e listagens
stepnumber=1,
firstnumber=1,
numberstyle=\tiny,
belowskip= 0.5cm,
numbersep= 0.2cm,
extendedchars=true,
breaklines=true,
frame=tb,
basicstyle=\footnotesize,
stringstyle=\ttfamily,
showstringspaces=false
}

\usepackage{parskip}
\usepackage{longtable}
\usepackage[toc]{glossaries}
%\makeglossaries

\hypersetup{
    colorlinks = true,
    citecolor = black,
    filecolor = black,
    linkcolor = black,
    urlcolor = black,
}

\usepackage[toc]{glossaries}
%\makeglossaries

\newcommand{\tituloMonografia}{Software para análise de variantes genômicas nos genes BRCA1 e BRCA2 a partir de dados obtidos em NGS}

\newcommand{\nomeAluno}{Mário M. T. B. de Rezende}

\newcommand{\nomeSupervisor}{Euclides Matheucci Jr.}

\newcommand{\nomeEmpresa}{QGene Ind. Com. de Equipamentos Para Laboratórios. Ltda.}

\renewcommand*{\lstlistlistingname}{Lista de Listagens}

\hyphenation{}

% abbreviations: coloque em ordem alfabética
\newacronym{ans}{ANS}{Agência Nacional de Saúde}
\newacronym{bcc}{BCC}{Bacharelado em Ciências da Computação}
\newacronym{ct}{CT}{\textbf{casos de teste}}
\newacronym{cep}{CEP}{Código de Endereçamento Postal}
\newacronym{fmrp}{FMRP}{Faculdade de Medicina de Ribeirão Preto}
\newacronym{icmc}{ICMC}{Instituto de Ciências Matemáticas e de Computação}
\newacronym{mvc}{MVC}{\textit{Model-view-controller}}
\newacronym{ngs}{NGS}{\textit{Next Generation Sequencing}}
\newacronym{ror}{RoR}{Ruby on Rails}
\newacronym{sp}{SP}{São Paulo}
\newacronym{tdd}{TDD}{\textit{Test Driven Development}}
\newacronym{usp}{USP}{Universidade de São Paulo}


\begin{document}
\pagenumbering{roman}

\begin{titlepage}   
	\pagestyle{empty} % no headers in this environment

  	% title page
	\begin{center} 
    \begin{minipage}[c]{12cm}
    \begin{center}
      \vspace{.35\textheight}
      \hrulefill\\
      \vspace{.5cm} {\Large \textcolor{blue}{\tituloMonografia}}\\
      \vspace{1.3cm}
      \textbf{\nomeAluno}\\
      \vspace{.5cm}
      \hrulefill\\
    \vspace{5cm}
    \includegraphics[width=5cm]{img/logoICMC.png}
    \end{center}
    \end{minipage}
  \end{center} 
  

  \cleardoublepage
 

  \vspace*{3cm}
  \begin{center}
    {\huge\sf \textcolor{blue}{\tituloMonografia}} \\
    \vspace*{2cm}
    {\bf \nomeAluno} \\
    \vspace*{2cm}
    \emph{Supervisor:}  {\nomeSupervisor}\\
    \emph{Empresa:}  {\nomeEmpresa}
  \end{center}
  \vspace*{3cm}

  \begin{flushright}
    \begin{minipage}{10cm}
      Monografia de conclusão de curso apresentada ao
Instituto de Ciências Matemáticas e de Computação da Universidade de São Paulo para obtenção do título de Bacharel em
Ciências de Computação.
    \end{minipage}
  \end{flushright}

  \vspace*{2cm}
  \begin{center}
    \textbf{USP - São Carlos \\ Maio de 2014}
  \end{center}
  
  \cleardoublepage  
  
\end{titlepage}

\thispagestyle{plain}  
\setcounter{page}{1}

\chapter*{Dedicatória}

Dedico minha monografia aos meus pais, que sempre me apoiaram a perseguir meus sonhos.

\chapter*{Agradecimentos}

Agradeço a todas as pessoas que fizeram parte de minha vida até o presente momento.


%%%%%%%%%%%%%%%%%%%%%%%%%%

\input resumo.tex

\noindent \textbf{Palavras-chaves:} rails, TDD, bioinformática, DNA.

\tableofcontents
%\listoffigures
%\listoftables
\lstlistoflistings
\printglossary[title=Lista de Termos,toctitle=Termos e Abreviaturas]

\onehalfspacing

\mainmatter

\renewcommand{\chaptermark}[1]{%

\markboth{\chaptername
\ \thechapter.\ #1}{}}  %

\renewcommand{\sectionmark}[1]{%
 \markright{\thesection.\ #1}}

%\input ...
\input introducao.tex
\input planejamento.tex
\input desenvolvimento.tex
\input conclusao.tex

%Observação 1: É obrigatório que a monografia tenha uma lista de referências que deve estar contida neste item.
%Observação 2: As referências devem estar em ordem alfabética pelo sobrenome do primeiro autor.
%Observação 3: Todos os documentos referenciados nesse item (livros, artigos, relatórios técnicos, sites, entre outros) deverão ter sido citados no texto.
%Observação 4: Todos as citações no decorrer do texto deverão ser listadas neste item.
%Observação 5: É obrigatório que a lista de referências esteja de acordo com a norma NBR-6023/2002 para referências bibliográficas.

\renewcommand{\bibname}{Referências}
%\printbibliography[heading=bibintoc]
\bibliographystyle{plain} 
\bibliography{referencias}{}

% Se precisar de apêndice, use a seção abaixo
%\appendix
%\renewcommand\appendixname{Apêndice}
%\renewcommand\chaptername{Apêndice}

\end{document}
