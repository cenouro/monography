\chapter{Planejamento do trabalho}
\label{chap:atividadesPlanejadas}

\section{Atividades planejadas para o estágio}

A atividade planejada é a implementação de um \textit{pipeline} para análise de dados de \gls{ngs} e gerar laudos automaticamente. Os equipamentos de \gls{ngs}
~\cite{NGEN} produzem \textit{output} em um formato padrão que pode facilmente passar por um processo de \textit{parsing} 
por outras ferramentas. O objetivo deste sistema é que ele faça processamentos de tal \textit{output} com o intuito de melhorar a 
confiabilidade do sequenciamento (eliminar falsos negativos, especialmente) 
e melhorar a vizualização dos dados para ajudar o usuário a tomar decisões pertinentes sobre a emissão do laudo.

Como a bioinformática voltada para análises genéticas é uma área nova para mim, o desenvolvimento do sistema computacional não contou com um cronograma.
Ao invés disso, o desenvolvimento e o estudo sobre áreas da bioinformática que não domino (áreas pertinentes a biologia e genética) são realizados de 
forma altamente dinâmica, à medida que surge a necessidade.

A única atividade que foi planejada foi a minha participação no curso de vereão de bioinformática ministrado na \gls{fmrp}~\cite{BIOC}. O curso foi bem proveitoso
e pude melhorar em muito minha comunicação com a equipe da empresa e meu domínio de tal área. O curso, além de abordar aspectos teóricos de genética, também
continha aulas práticas onde ensinaram a utilizar a linguagem R e o \textit{framework} Bioconductor, um \textit{framework} com vários métodos para
realizar \textit{parsing} de arquivos de sequenciamento, analisar estatísticas e verificar mutações. Esse \textit{framework} é muito poderoso, porém não
está sendo utilizado no desenvolvimento do sistema de análise pois é mais voltado para análises de grande porte (análises de populações), o que contrasta com
os requisitos de análise clínica. Esse contraste não justifica, até o momento, a utilização de mais uma linguagem e mais um \textit{framework} no sistema, pois
pode prejudicar o tempo de treinamento de novos funcionários e/ou a manutenibilidade do sistema.

%Discorra sobre quais atividades foram originariamente planejadas para o estágio, discutindo a quais problemas/objetivos estariam associados.

%Você pode fazer uso de tabelas, por exemplo para sumarizar o cronograma de atividades planejados. Quando você usar uma tabela (ou figura, ou quadro, ou listagem), é obrigatório fazer referência a ela quando seu contéudo é explicado no texto, como por exemplo: ver Tabela~\ref{tab:tab1}.

%  \begin{table}[h!]
%\begin{center}
%    \caption [Atividades Planejadas]
%  {Atividades Planejadas}\label{tab:tab1}
%
%    \begin{tabular}{llp{7cm}} \hline 
%
%\hline
%    \textbf{Início}    & \textbf{Fim} &  \textbf{Descrição}                             \\ 
%    \hline
%    01/02/2014       & 15/02/2014                & Atividade 1: Treinamento \\
%    15/02/2014       & 28/02/2014                & Atividade 2: Treinamento\\ 
%    01/03/2014       & 30/03/2014                & Atividade 3: Projeto\\ 
%    01/03/2014       & 30/03/2014                & Atividade 4: Desenvolvimento\\ 
%    01/04/2014       & 30/04/2014                & Atividade 5: Avaliação\\ \hline
%
% \hline
%    \end{tabular}
%\end{center}
%  \end{table}

\section{Treinamentos planejados para o estágio}

O único treinamento planejado foi a participação no curso da \gls{fmrp}. Demais treinamentos são realizados quando se fazem necessários.

Além de cursos, meu chefe está sempre disponível para que eu possa tirar dúvidas sobre os assuntos de genética e, inclusive, me convida
a participar de eventuais palestras por ele ministradas.
%Discorra sobre treinamentos previstos quando do início do estágio.

%Quando citar os treinamentos planejados, a duração de cada um pode ser apresentada em uma tabela, como ilustrado na Tabela~\ref{tab:tab2}.
%
%  \begin{table}[!ht]
%\begin{center}
%    \caption [Treinamentos Planejados]
%  {Treinamentos Planejados}\label{tab:tab2}
%
%    \begin{tabular}{llp{7cm}}
%    \hline 
%
%    \hline
%    \textbf{Início}    & \textbf{Fim} &  \textbf{Descrição}                             \\ 
%    \hline
%    01/02/2014       & 15/02/2014                & Treinamento em Ruby on Rails \\
%    15/02/2014       & 28/02/2014                & Treinamento em SOA\\
% 
%    \hline
%    \end{tabular}
%\end{center}
%  \end{table}
