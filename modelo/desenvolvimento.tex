\chapter{Desenvolvimento do trabalho}
\label{chap:atividadesRealizadas}


\section{Atividades realizadas}

Nesta seção são descritas as atividades que realizei durante o estágio. Para as duas atividades de desenvolvimento de software descritas abaixo,
a dinâmica de trabalho empregada foi a mesma: \gls{tdd}. Essa metodologia foi escolhida pois facilita em muito a comunicação com os solicitantes
dos softwares e é extremamente útil para projetos cujos requisitos são alterados constantemente.

\subsection{Software para análise dos genes BRCA1 e BRCA2}

Os genes BRCA1 e BRCA2 são responsáveis, entre outras coisas, pela capacidade da célula de corrigir possíveis erros em
seu DNA. Por esse motivo, mutações patogênicas em tais genes tem um grande impacto na vida do paciente. Uma célula que
não consegue se corrigir tem alta probabilidade de gerar células cancerosas ao se reproduzir.

A \gls{ans} emitiu recentemente uma resolução~\cite{RolANS} que obriga todos os planos de saúde no Brasil a darem cobertura para
sequenciamentos de BRCA1 e BRCA2, entre outras doenças. Tais sequenciamentos devem ser solicitados por médicos, e podem ser solicitados
tanto por medida preventiva como para confirmar diagnósticos.

A atividade planejada é um sistema web com a finalidade de facilitar a requisição de tais exames e emissão de laudo preliminar.
Dessa forma, o sistema ajudaria e disponibilizaria informações para que o médico escolha o melhor exame pertinente a cada doença (a princípio,
relacionada ao BRCA1 e BRCA2) e solicite o exame. E, no resultado, teria informações pertinentes para construir e emitir o laudo final.

Outro requisito do sistema é que ele faça o processamento de forma automática dos dados após o sequenciamento. No \gls{ngs}, utilizam-se
reagentes químicos e enzimáticos para isolar e marcar as áreas do DNA que devem ser sequenciadas. Após o sequenciamento, esses reagentes, se
não forem eliminados do \textit{output} da máquina de sequenciamento, podem causar ruídos na etapa de análise dos dados. Tais ruídos
são causados porque alguns dos reagentes são trechos pequenos de DNA que, por construção, são idênticos ao DNA de referência. Se o DNA
sequenciado apresenta mutações especificamente nas regiões dos reagentes, a máquina acusará que em alguns ciclos do sequenciamento detectou
tais mutações (no DNA original) e em outros não detectou (no reagente). Isso diminui a confiabilidade do sequenciamento e influencia
a análise do DNA para a emissão do laudo.

Por fim, ao longo prazo, o sistema deve armazenar os resultados dos sequenciamentos para que a empresa possa minerar dados e possivelmente
evitar que um mesmo paciente realize o mesmo sequenciamento mais de uma vez (por exemplo, mesmo gene e doenças diferentes). Dados os requisitos 
de tal sistema, planejei ele em forma de \textit{pipeline} para receber solicitações de exames, fazer o processamento dos dados do 
sequenciamento e armazenar no banco de dados. O módulo de mineração de dados será um módulo futuro.

\subsection{Programação do robô de pipetação}

Fui solicitado a programar um robô~\cite{Arise} para pipetar reagentes. A empresa vende \textit{kits}, que são misturas de compostos
químicos e enzimáticos (por exemplo, para realizar testes de paternidade). Os compostos precisam ser pipetados e misturados em volumes precisos.
Atualmente, a empresa conta com funcionários no laboratório que pipetam manualmente tais \textit{kits}, mas deseja utilizar o robô pois tal tarefa
é muito repetitiva.

O robô (um braço mecânico) é programado através de um software proprietário de fácil utilização. Basta indicar o tubo de origem, o tubo de destino e a quantitdade
de cada reagente que deve ser pipetado. A sequência de comandos pode ser salva, podendo ser repetida mais vezes.

\subsection{Reimplementação de sistema legado}

A atividade na qual estou trabalhando atualmente é a reimplementação de um sistema legado. O sistema é responsável por cadastrar clientes e 
animais e fazer o acompanhamento e marcha analítica dos exames de paternidade. O objetivo da reimplementação é utilizar tecnologias mais robustas para o sistema
(PostgreSQL, por exemplo), disponibilizar o sistema como um sistema web e utilizar o serviço de \textit{storage} da Amazon.

%Descreva quais atividades foram realizadas durante o projeto. 

%Para cada atividade realizada, discuta também qual foi a dinâmica de trabalho --- por exemplo, se foi utilizada a metodologia SCRUM ou algo similar.

%Se apropriado, você apresentar algoritmos ou códigos, como o ilustrado na Listagem~\ref{src:code1}, mas deve explicar o seu funcionamento no texto.

%\begin{lstlisting}[
%  language=C,
%  caption=Exemplo de código,
%  label=src:code1
%]
%#include <stdio.h>
%int main(){ 
%    int i=0, j=1;
%    printf("i:%d j:%d\n",i,j);
%    return;
%}
%\end{lstlisting}
%
%Você também pode fazer uso de figuras (como a Figura~\ref{fig:fig1}), mas deve explicar a figura no texto.
%
%\begin{figure}[!ht]
%  \rule[1ex]{\textwidth}{0.25pt}
%  \centering\includegraphics[width=0.25\textwidth]{img/logoICMC.png}
%  \caption [Exemplo de figura]
%  {Exemplo de figura}\label{fig:fig1}
%  \rule[1ex]{\textwidth}{0.25pt}
%\end{figure}


\section{Problemas resolvidos}
%Descreva quais problemas foram resolvidos. 
Nesta seção, são descritos problemas enfrentados em cada atividade e que consegui resolver.

\subsection{Software para análise dos genes BRCA1 e BRCA2}

Os dados gerados pelo \gls{ngs} (padronizados segundo ~\cite{SamTools}) precisam ser processados antes de serem analisados, com a finalidade de aumentar a qualidade
dos resultados (como descrito na seção anterior).
Até o presente momento, o sistema ainda está em fase de prototipação. \gls{ngs} é uma forma de sequenciamento
recente e ainda está sendo estudada pela empresa. Dessa forma, os protótipos desenvolvidos estão ajudando a entender melhor os conceitos, os gráficos
gerados pela \textit{cloud} do fabricante, e os formatos de arquivos para representar dados de sequenciamento. Esse é um passo muito importante para que
o laudo emitido pelo médico seja baseado nas informações mais fidedignas possíveis.

Um artifício utilizado pela empresa para avaliar se os protótipos estão produzindo os dados esperados foi a análise de cinco pacientes.
O conjunto de pacientes que participaram foi formado por uma mãe, que já teve câncer de ovários, e suas quatro filhas, que, até então, estão saudáveis;
contudo, espera-se que venham a ter risco mais elevado de desenvolverem câncer de ovários dado o histórico de sua mãe.

Os protótipos detectaram a mutação na mãe (uma inserção, que muda o quadro de leitura dos aminoácidos, o que a torna é muito patogênica). Nas filhas, 
nada foi detectado, apesar disso ser um resultado inesperado.

\subsection{Programação do robô de pipetação}

O problema que o robô apresenta é o erro na pipetação. A ponteira de pipeta utilizada pelo robô armazena uma quantidade muito pequena de líquido (cinquenta
microlitros). Isso gerou a dificuldade de pipetar alguns reagentes que continham substâncias viscosas em sua composição, como o glicerol. Líquidos
viscosos podem formar uma bolha na ponta da pipeta e/ou parte do líquido pode ficar grudada dentro da pipeta.

Para eliminar esse problema, o que fiz foi pipetar os solventes antes. Após os solventes estarem no tubo de destino, programei o robô para pipetar os
reagentes viscosos e, logo em seguida, realizar uma mistura. O comando de mistura faz com que o robô pipete e solte os reagentes do tubo repetidamente.
Assim, os reagentes que não eram expelidos, apesar de continuarem presos na pipeta, eventualmente eram diluídos o suficientemente para serem expelidos
sem perda de precisão.

\subsection{Reimplementação de sistema legado}

Essa atividade não apresentou (até então) problemas, uma vez que é apenas uma reimplementação de um sistema ao qual tenho acesso para
poder realizar engenharia reversa.

\section{Técnicas, métodos e tecnologias envolvidas}

Aqui, são apresentados alguns detalhes sobre o desenvolvimento do sistema.
As ferramentas usadas no desenvolvimento são bem documentadas e apresentam uma abundância de
exemplos na internet. A descrição detalhada de cada ferramenta está fora do escopo deste trabalho.

\subsection{\gls{ngs}}

\gls{ngs}~\cite{NGEN} é uma técnica para sequenciamento de DNA feita por leituras que utilizam um \textit{laser} e alguns reagentes que emitem
diferentes comprimentos de onda de luz. Assim sendo, após um ciclo de leitura, o equipamento tem uma imagem com cores que representam
os diferentes nucleotídeos do DNA sequenciado, e aplica algoritmos de processamento de imagem para mapear tal imagem em um arquivo
texto para fácil manipulação.

\gls{ngs} é uma técnica um tanto quanto recente, e pode variar em alguns aspectos dependendo do fabricante do equipamento. O nome \gls{ngs}
é usado para representar os sequenciamentos que podem fazer leituras de trechos grandes de DNA.

Como toda técnica, essa possui certas desvantagens. O resultado é muito preciso, porém existem técnicas que produzem resultados mais fidedignos
quando o trecho de DNA sequenciado é pequeno. Além disso, essa técnica, como dito antes, produz resultados viesados pois utiliza-se de reagentes
que podem introduzir falsos negativos. Por isso, o processo deve ser muito bem acompanhado e os dados produzidos precisam ser tratados após a fase de
sequenciamento.

\subsection{BaseSpace}

BaseSpace~\cite{BS} é a \textit{cloud} do fabricante do equipamento 
de \gls{ngs} sendo
utilizado pela empresa atualmente. Após o término do sequenciamento,
o equipamento realiza o alinhamento com o DNA de referência e envia os dados
brutos e dados de alinhamento para a \textit{cloud}. Essa \textit{cloud}
possui alguns aplicativos web que podem ser utilizados para analisar
os dados, mas tais análises são um tanto quanto genéricas (por exemplo, 
não consideram a possibilidade de ruídos devido aos reagentes).

De qualquer forma, a \textit{cloud} ainda é utilizada pois armazena os dados
e, eventualmente, a empresa cogita a possibilidade de disponibilizar o sistema
que estou desenvolvendo nessa \textit{cloud} e cobrar um valor módico por análise.

\subsection{Ruby on Rails}

O \textit{framework} usado para a implementação do sistema é \gls{ror}~\cite{RoR}.
\gls{ror} é um \textit{framework} poderoso para construção de aplicativos web. \gls{ror} implementa de forma simples e elegante
o modelo \gls{mvc}.

\gls{ror} foi escolhido por alguns motivos. É um \textit{framework} com alto poder de prototipação, abstrai totalmente
a parte de banco de dados (SQL só é necessário para otimizar certas consultas), conta com uma comunidade muito ativa,
pode ser facilmente configurado para utilizar os serviços de \textit{storage} da Amazon, e é um \textit{framework} que
incentiva o desenvolvedor a escrever casos de teste a fim de ter maior segurança durante a codificação e refatoração do código.

\subsection{RSpec}

RSpec~\cite{RSpec} é uma biblioteca (no contexto de \gls{ror}, também chamadas de \textit{gemas}) projetada para a descrição e implementação de
\gls{ct}. A sintaxe do RSpec é muito parecida com a sintaxe da linguagem Ruby ao mesmo tempo que se parece muito com a língua inglesa,
como exemplificado na Listagem~\ref{src:ex1}. Tal listagem exemplifica também a estrutura básica dos casos de teste: \textbf{describe}, que
define o sujeito do caso de teste (no caso, um objeto da classe \textbf{Pdf}), \textbf{specify}, que é um caso de teste, e, dentro do caso de teste,
as expectativas (\textbf{should\_not} e \textbf{expect}), que, em conjunto com os \textit{matchers} (\textbf{be\_a\_new\_record}, \textbf{raise\_error}),
verificam o comportamento do sistema. Em outras palavras, a listagem exibe um caso de teste que testa um \textbf{pdf} que possui título. Tal
\textbf{pdf} deve estar salvo no banco de dados e, quando tentarmos inserí-lo novamente no banco de dados, o sistema deve levantar uma exceção
indicando que o título do \textbf{pdf} já está sendo utilizado por outra tupla.
Isso faz com que os \gls{ct} sejam muito descritivos e concisos, tanto para o desenvolvedor quanto para o cliente, diminuindo a dificuldade
de comunicação.

É possível escrever \gls{ct} para os modelos, controladores, e até mesmo para as \textit{views} do sistema. Com a disponibilidade de uma gema
tão robusta e bem projetada, tornou-se uma convenção a utilização de \gls{tdd} no desenvolvimento de sistemas escritos em \gls{ror}.

\begin{lstlisting}[
  language=Ruby,
  caption=Exemplo ilustrativo da sintaxe do RSpec (sem outras gemas),
  label=src:ex1
]

describe Pdf do
	specify "when title is not unique" do
		pdf.should_not be_a_new_record
		expect { Pdf.create!(pdf.attributes) }.to raise_error ActiveRecord::RecordInvalid
	end
end

\end{lstlisting}

\subsubsection{Shoulda e Capybara}

Shoulda~\cite{Shoulda} e Capybara~\cite{Capybara} são gemas que incrementam a funcionalidade do RSpec.

Shoulda possui métodos para testar controladores ou modelos com apenas uma linha de código. Por exemplo, existe um método (Listagem~\ref{src:belong_to})
que verifica se a tabela correspondente
a um modelo foi criada no banco de dados, se os campos da tabela estão corretos, e se as chaves estrangeiras estão referenciando as chaves primárias nas 
tabelas corretas. Essa gema não foi utilizada no início do estágio, mas quando foi adotada no projeto, o tamanho dos arquivos de \gls{ct} foi reduzido 
consideravelmente. Isso foi muito importante, pois os protótipos não contam com um documento de requisitos, apenas com os \gls{ct}, e pode-se 
configurar o Rspec para que ele solte um \textit{output} em formato de documentação. Deixar a implementação dos \gls{ct} concisa reduziu em muito 
o tamanho da documentação gerada pelo Rspec.

\begin{lstlisting}[
  language=Ruby,
  caption=Exemplo de Shoulda,
  label=src:belong_to
]

describe Post do
	it { should belong_to :user }
end

\end{lstlisting}

Capybara é uma gema com métodos para se realizar testes de integração. Testes de integração são testes que simulam um usuário utilizando o sistema para
determinadas tarefas. Por exemplo, alguns métodos da gema (Listagem~\ref{src:fill_form}) permitem simular um usuário que visita uma página,
preenche um formulário, e pressiona o botão de submissão do formulário. Se, após tudo isso, o usuário for redirecionado para a página correta e os dados
forem salvos no banco de dados, então o aplicativo passa nesse caso de teste.

\begin{lstlisting}[
  language=Ruby,
  caption=Exemplo de métodos do Capybara,
  label=src:fill_form
]

describe "signup" do
	visit signup_path
	fill_in "Name", with: "Example User"
	fill_in "Email", with: "user@example.com"
	fill_in "Password", with: "foobar"
	fill_in "Confirmation", with: "foobar"
	
	# expectations in plain Rspec or Shoulda ...
end

\end{lstlisting}

\subsection{\gls{tdd}}

\gls{tdd} é uma metodologia de desenvolvimento onde os \gls{ct} são escritos antes da aplicação ser implementada. Tais casos de teste guiam
o processo de desenvolvimento. Certos \gls{ct} são escritos consultando-se o cliente para definir o comportamento do sistema, de tal forma que 
tais casos de teste possam ser vistos como uma forma de documento de requisitos extremamente flexível do sistema. Os demais \gls{ct}
são escrito para evitar/detectar \textit{bugs} e proporcionar maior segurança e confiança ao desenvolvedor quando manutenções ou alterações forem necessárias.

O aspecto de evitar/detectar \textit{bugs} é extremamente importante uma vez que \gls{ror} implementa certas medidas de segurança. Detectar e 
corrigir \textit{bugs} nos modelos são tarefas simples, pois os modelos implementam apenas as lógicas de negócios e tabelas do banco de dados. Mas tais tarefas
podem ser muito complicadas nos controladores, que são responsáveis pelo comportamento do sistema como um todo e também implementam diretivas de segurança.
Uma diretiva de segurança em particular evita o problema de \textit{mass assignment}, e caso o desenvolvedor não configure tal diretiva, pode acabar com
dados espúrios no banco de dados. Tal diretiva é silenciosa, ou seja, não emite erros/exceções caso o sistema receba um ataque de \textit{mass assignment}.

%Descreva os métodos, técnicas  e tecnologias
%envolvidos  ou que  foram utilizados  para a  condução das  atividades
%durante  o estágio.  Por exemplo:  (i)  no caso  dos métodos,  pode-se
%apresentar XP (eXtreme Programming) e SCRUM, métodos empregados para o
%teste de  sistemas, entre  outros; (ii) No  caso de  técnicas, pode-se
%descrever técnicas da UML, técnicas de teste, entre outros; e (iii) no
%caso  de  tecnologias,  pode-se descrever  ferramentas  (por  exemplo,
%Spring, Hibernate,  Struts, entre  outros), linguagens  de programação
%(por exemplo, Java), padrões de projeto utilizadas, entre outros.
%
%Faça  referências bibliográficas  atuais e  de fontes
%relevantes  (evite  sites,  privilegie   livros  ou  artigos);  mostre
%diferentes  tecnologias e  faça comparações,  quando for  o caso.  Não
%esquecer de citar a fonte corretamente, por exemplo~\cite{MichettiJavaMagazine2013}.

\section{Impacto}

A respeito do sistema para análise, os profissionais terão em mãos uma ferramente valiosa para ajudá-los a analisar os dados
do sequenciamento. Utilizando-se a ferramenta, espera-se que os prognósticos sejam mais precisos e possam ser usados para
previnir casos clínicos antes que estes ocorram.

Sobre as demais atividades, espera-se que os custos operacionais da empresa diminuam e tal redução possa ser repassada para os clientes.

%Quais foram as pessoas/entidades afetadas por esses resultados? Quem são  as pessoas/entidades que potencialmente serão afetadas?

%Fale sobre a relevância da solução do(s) problema(s) para a empresa e seu(s) cliente(s).

%Você pode omitir informações sigilosas.

%Discuta também sobre a relevância para sua formação a  participação na solução do(s) problema(s).

\section{Problemas não resolvidos}

Não existe versão final do sistema de análise ainda. Apenas protótipos que estão sendo usados para explorar possíveis funcionalidades e outros aspectos do problema.

Sobre o robô, apesar de resolver o problema de reagentes viscosos, ainda estamos verificando se é viável sua utilização, uma vez que
existe uma variância na quantidade de reagentes pipetados devido aos erros de medições do equipamento. Esse erro é especificado em seu manual,
mas a empresa não sabe ainda se tal erro pode ser tolerado nas reações, já que as reações devem ser muito precisas. Como exemplo, a concentração de enzimas
nas reações deve ser da ordem de picomols; concentrações mais grosseiras ou finas podem influenciar na cinética enzimática, proporcionando erros nas reações.
%Apresente quais os problemas que não puderam ser resolvidos -- e justifique.
