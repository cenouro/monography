\chapter{Introdução}
\label{chap:intro}
\pagenumbering{arabic}  %%% deixe aqui para começar no Nro 1!!!!!!!

\section{Sobre a Empresa}
A QGene Ind. Com. de Equipamentos Para Laboratórios. Ltda. foi fundada em 21 de julho de 2006 e instalada 
na incubadora CEDIN (Centro de Desenvolvimento de Industrias Nascentes), do sistema de incubadoras da 
FIESP. A partir de 2009, a empresa foi transferida para nova sede situada na Rua Santa Cruz, 969, em São Carlos, 
para comportar o crescimento da empresa. A QGene surgiu como uma \textit{spin-off} da DNA Consult Genética e
Biotecnologia Ltda., empresa fundada há 12 anos e especializada em análises do DNA, com forte atuação em P\&D
(possui em seu portfólio três projetos PIPE-FAPESP). Entre janeiro e dezembro de 2007, a QGene abrigou um 
projeto PIPE-FAPESP relacionado com monitoramento e diagnóstico de vírus em camarões cultivados, tendo 
recebido cerca de R\$ 150.000,00 da FAPESP, para compra de insumos, equipamentos e bolsas.

A QGene atua na área de genética molecular, dedicando-se ao P\&D, produzindo e comercializando \textit{kits} para 
análise de DNA e soluções em saúde humana, que representem inovações tecnológicas e que substituem a 
importação de produtos.

A QGene é uma empresa de pequeno porte que goza de parcerias com diversas empresas, como a DNA Consult e MD Genetics.
Por ser uma empresa de pequeno porte, a QGene também está sempre procurando e atuando em nichos que não
eram o foco inicial da empresa, além de seus produtos consolidados há anos no mercado. Atualmente, a QGene
está produzindo soluções de bioinformática com foco clínico, uma vez que os softwares existentes, apesar
de robustos, não apresentam facilidade de uso para médicos e muitas vezes são softwares abrangentes
que não apresentam soluções clínicas.

A empresa valoriza muito seus funcionários e não mede gastos para aprimorá-los cada vez mais. A empresa entende
que os funcionários, sejam os contratados como os estagiários, podem ajudar a QGene a estar sempre se reinventando
como empresa e atacando novos nichos ou setores do mercado.

%Comente sobre a empresa: o setor de atuação, o porte, a missão e outras informações que você julgue necessárias, como por exemplo plano de carreira e premiações.

%Em toda a monografia, você pode fazer uso de  referencias bibliográficas para apresentar URLs \cite{WAI}), livros \cite{Dahl:1972} \cite{Hopcroft:1969} ou artigos utilizados no decorrer do estágio, publicados em periódicos \cite{Almorsy:2012} ou revistas \cite{MichettiJavaMagazine2013}.

\section{Sobre o Processo Seletivo}
O processo seletivo por qual passei foi apenas uma entrevista. Fui entrevistado por dois sócios da empresa. Um deles
possui doutorado na área de biologia molecular, e o outro sócio trabalhava no Banco do Brasil, no cargo de presidência,
mas agora está aposentado. A entrevista não continha perguntas técnicas sobre computação, especialmente pois a QGene não
possuía, até então, um setor de TI. O que os entrevistadores avaliaram foi algo mais subjetivo e de interesse deles:
capacidade de trabalho em equipe, interesse em trabalhar com uma área interdisciplinar (no caso, bioinformática),
capacidade de trabalhar de forma autônoma (ou seja, sem uma pessoa de TI me liderando), e proatividade.

%Comente sobre como foi o processo seletivo para admissão no estágio. Discuta como você se preparou para ele.

\section{Apresentação da monografia}

Os próximos capítulos irão retratar de forma mais detalhada as atividades realizadas por mim durante o estágio.

No Capítulo~\ref{chap:atividadesPlanejadas}, descrevo brevemente como foi o planejamento das atividades. É apresentado também
como foi meu treinamento.

No Capítulo~\ref{chap:atividadesRealizadas}, apresento algumas ferramentas que utilizo para desenvolver o sistema. Também explico brevemente os motivos
que me levaram a adotar a metodologia de \gls{tdd}.

No último capítulo, são levantadas as conclusões sobre o estágio e a relação com 
o curso de graduação, indicando melhorias e perspectivas para o futuro, tanto para o curso 
quanto para o trabalho.
