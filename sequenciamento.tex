Aqui, descrevo mais detalhadamente o processo de sequenciamento do DNA.
Para entender melhor sobre o sequenciamento, conversei bastante com a equipe
da empresa e tive a oportunidade de participar do curso de verão \textit{10th Summer Bioinformatics Course}, organizado
pela FMRP-USP e renomado internacionalmente. O curso abordou algumas técnicas para análise
de variantes utilizando-se a linguagem R e formas de projetar o experimento a fim de melhorar a qualidade do
sequenciamento.

%%sequenciamento
O primeiro passo do experimento é a colheta da amostra do paciente. Após ser colhetada, a amostra
de DNA é amplificada. Amplificar significa fazer várias cópias do \textbf{trecho de interesse}.
Isso é feito com o uso de marcadores. Tais marcadores são trechos de DNA que se alinham com o
início e com o fim das regiões de interesse e, dessa forma, possibilitam que a polimeraze (enzima
responsável pela replicação do DNA) replique apenas o trecho de interesse. Demais trechos não são amplificados.

Ao término do processo de amplificação, teremos algumas fitas de DNA completas (da amostra original) e vários outros pedaços menores de DNA,
que são os trechos de interesse. A quantidade dos trechos de interesse é muito maior que o DNA original, o que permite que o sequenciamento
seja mais preciso. Além disso, antes do sequenciamento, as fitas completas de DNA são removidas através de um crivo (elas possuem peso molecular
muito maior que os trechos menores).

O equipamento realiza o sequenciamento. É um processo bem complicado, explicado aqui brevemente. O equipamento usa um laser para sequienciar os trechos
menores de DNA, e tal laser gera imagens. Tais imagens então passam por um algoritmo de processento de imagens para gerar o output. Esse processo é
repetido várias vezes, o que significa que cada trecho é sequenciado várias vezes (tanto pela repetição do processo, quanto pelo fato
do trecho ter sido amplificado várias vezes na etapa de amplificação).

Ao fim do processo, temos várias \textit{strings} que representam os diversos sequenciamentos dos trechos. Cada \textit{string} é chamada de \textit{reads}.

%%alinhamento
Após o sequenciamento, os reads passam por três etapas. Primeiro, eles são alinhados com um DNA de referência. Isso é preciso pois o processo
de amplificação quebra a ordenação do DNA. Com o alinhamento, é possível recuperar tal estrutura para se analisar os resultados. Note que
a estrutura não é recuperada por completo. Suponha que temos um indivíduo heterozigoto que apresenta duas mutações em dois exons diferentes do gene
BRCA1, e suponha que o mesmo gene não apresente mutações no outro cromossomo. Após a amplificação, teríamos um exon normal e um mutante para cada um dos
dois exons. O sequenciamento detecta os mutantes e os exons normais, porém não é possível reconstruir a estrutura original. As duas mutações poderiam estar
no mesmo gene ou uma mutação estaria em um dos genes, e a outra, no outro gene. Isso é um exemplo de perda de informação que não é nem um pouco raro.

O segundo passo é podar as sequências dos marcadores. Esse passo aumenta drasticamente a qualidade de leitura das mutações. Muitos reads se alinham na mesma
região do DNA, porém os reads são projetados para serem intercalados. Por serem intercalados, podemos ter situações onde uma mutação detectada em um read
coincide com a região de marcador de outro read. O marcador não possui a mutação (pois é projetado com base no DNA referência), logo teremos um read acusando
mutação e outro read não apresentando mutação. Se eliminarmos as regiões dos marcadores, o read com mutação continua apresentando-a (pois a mutação não se
encontra na região de marcador do read) e o outro read não apresentaria nada. Ou seja, se não eliminarmos os marcadores, a probabilidade da mutação é de
cinquenta porcento, e se eliminarmos os marcadores, a mutação tem probabilidade de cem porcento, nesse exemplo. Dessa forma, conseguimos eliminar um falso
negativo.

O terceiro passo é verificar a cobertura. Os reads perfeitamente alinhados entre si podem ser reads diferentes de um mesmo trecho de DNA, ou reads iguais
(o processo de sequenciamento não é capaz de diferenciar). Ou seja, podemos ter dois reads alinhados entre si numa mesma região, e que, na verdade, é apenas
um read contado duas vezes. Para eliminar esse erro, os reads são projetados para terem certo grau de intercalação. Ainda teremos reads alinhados entre si,
mas também teremos reads intercalados, o que nos garante uma maior qualidade de sequenciamento. Outra forma de amenizar o erro é verificar os reads com
algumas ferramentas de visualização de DNA. Nas ferramentas, podemos ver os reads. Muitos reads ficam alinhados entre si, mas esses reads podem apresentar
diferenças pequenas (quen ão caracterizam uma mutação) de nucleotídeos. 
Dois reads alinhados entre si mas que apresentem divergência de nucleotídeos são necessariamente diferentes. Esses tratamentos aumentam a cobertura,
que é uma métrica para avaliar a qualidade do sequenciamento da região.

%%variantes
Após o alinhamento, é gerado um arquivo (um arquivo texto padronizado pela 1000 Genomes) com as variantes. O alinhamento não é feito de forma exata. 
O alinhamento usa algoritmo de alinhamento de strings que permitem certa flexibilidade. Por exemplo, ``banana" e ``canana" não são strings exatamente 
iguais, mas ainda é possível alinhar grande parte delas. Fazendo isso, chegaríamos à conclusão de que a letra inicial ('b' ou 'c', dependendo de qual 
string é a referência) é uma mutação.

O arquivo de mutações agrega outras informações, como qualidade de leitura, código da região, posição da mutação, entre outras coisas. Esse arquivo também
pode passar por um processo chamado de \textbf{processo de anotação}, onde cada mutação é verificada em bancos de dados renomados e as informações
contidas nesses bancos de dados são adicionadas ao arquivo de mutações.

Feito isso, o último passo é gerar o laudo.
