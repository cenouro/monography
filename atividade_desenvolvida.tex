Inicialmente, o único projeto era o software para ajudar nas análises das variantes e gerar o laudo. Posteriormente, outro projeto começou a ser desenvolvido
em paralelo.

Os projetos sendo desenvolvidos são os seguintes:

1: Pipeline para análise de dados de sequenciamento de próxima geração. Os equipamentos de sequenciamento de próxima geração produzem output em 
um formato padrão que pode ser facilmente passar por um processo de \textit{parsing} por outras ferramentas. O objetivo deste sistema é que ele 
faça processamentos de tal output com o intuito de melhorar a confiabilidade do sequenciamento (eliminar falsos negativos, especialmente) 
e melhorar a visibilidade do usuário.

2: Reimplementação de um sistema de controle de teste de paternidade de animais e humanos. O sistema que está sendo utilizado atualmente usa 
tecnologias defasadas (Firebird como banco de dados, banco de dados local e sem redundância, etc). O objetivo é implementar um sistema semelhante 
com tecnologias bem sedimentadas (Postgresql, storage da Amazon ou outra cloud, sistema web para facilitar o acesso, entre outras) e migrar 
os dados posteriormente para o novo sistema.

Os sistemas estão sendo desenvolvidos no framework Ruby on Rails (RoR). 


