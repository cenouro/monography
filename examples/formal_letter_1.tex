%%%%%%%%%%%%%%%%%%%%%%%%%%%%%%%%%%%%%%%%%
% Thin Formal Letter
% LaTeX Template
% Version 1.11 (8/12/12)
%
% This template has been downloaded from:
% http://www.LaTeXTemplates.com
%
% Original author:
% WikiBooks (http://en.wikibooks.org/wiki/LaTeX/Letters)
%
% License:
% CC BY-NC-SA 3.0 (http://creativecommons.org/licenses/by-nc-sa/3.0/)
%
%%%%%%%%%%%%%%%%%%%%%%%%%%%%%%%%%%%%%%%%%

%----------------------------------------------------------------------------------------
%	DOCUMENT CONFIGURATIONS
%----------------------------------------------------------------------------------------

\documentclass{letter}

\usepackage[portuguese, brazilian]{babel}
\usepackage[utf8]{inputenc}

% Adjust margins for aesthetics
\addtolength{\voffset}{-0.5in}
\addtolength{\hoffset}{-0.3in}
\addtolength{\textheight}{2cm}

%\longindentation=0pt % Un-commenting this line will push the closing "Sincerely," to the left of the page

%----------------------------------------------------------------------------------------
%	YOUR NAME & ADDRESS SECTION
%----------------------------------------------------------------------------------------

\signature{Mario M. T. B. de Rezende} % Your name for the signature at the bottom

\address{Rua Achille Bassi \\ São Carlos, SP 2717 \\ (16) 98837-3179} % Your address and phone number

%----------------------------------------------------------------------------------------

\begin{document}

%----------------------------------------------------------------------------------------
%	ADDRESSEE SECTION
%----------------------------------------------------------------------------------------

\begin{letter}{João Alfredo Miranda Botelho \\ QGene Ind. e Com. de equipamentos para Lab. Ltda. \\ Rua Santa Cruz \\ São Carlos, SP 969} % Name/title of the addressee

%----------------------------------------------------------------------------------------
%	LETTER CONTENT SECTION
%----------------------------------------------------------------------------------------

\opening{\textbf{Prezado João,}}

Venho por esta informá-lo sobre os acontecimentos de nossa viagem para Brasília.

A platéia era formada por volta de trinta pessoas (médicos e enfermeiros, em sua maioria) e pela \textbf{Cristiane Reis}
(uma moça do setor de administração, se me lembro bem). Os médicos e enfermeiros pertecentes à platéia trabalhavam
em cargos com certo grau de burocracia, tais como avaliar se o plano de saúde deveria liberar uma requisição
de sequenciamento ou se isso deveria ser negado e lidado em posteriores instâncias.

A palestra ministrada pelo Euclides foi bem aceita e acrescentou conhecimento aos espectadores. Eles puderam entender melhor
sobre o funcionamento, resultados e até mesmo preços e práticas antiéticas que estão sendo exercitadas por outros
laboratórios. Foi inclusive preparada uma provinha ao final do curso para avaliar a absorção e presença dos espectadores (acho que foi
o Euclides quem preparou tal provinha).

Ademais, pudemos ter um contato melhor com os problemas que estão sendo enfrentados pela Cassi e outros planos de saúde.
Tais problemas serão o assunto principal desta carta.

A Cassi é uma entidade que precisa disponibilizar cobertura para exames descritos por duas outras entidades: ANS e Tuss.
A ANS formulou um Rol que especifica os exames que devem ser cobertos pela Cassi. O Tuss é uma tabela que descreve também
os procedimentos que devem ser cobertos pela Cassi, com a diferença de que o Tuss também especifica um \textbf{código} para cada
tipo de procedimento. Esse código, ou identificador, é utilizado pela Cassi para emitir notas fiscais sobre preços, orçamentos,
entre outras coisas.

Existe outra entidade que especifica procedimentos: a CBHPM. Os procedimentos e códigos disponibilizados pela CBHPM também especificam
preços dos procedimentos e tal preço é melhor estruturado do que o Tuss (ficará mais claro adiante). Porém, a Cassi não é obrigada a
responder/utilizar os códigos descritos pela CBHPM.

Com essa visão geral, aqui estão os problemas e preocupações da Cassi:
\begin{itemize}
\item As três entidades envolvidas (Tuss, CBHPM e ANS) especificam procedimentos de formas superficiais, divergentes e contraditórias.
\item Tais contradições/divergências/superficialidades estão gerando \textbf{brechas legais} sobre o que a Cassi deve cobrir ou não, e tais
brechas estão sendo usadas por alguns pacientes.
\item Tais brechas são pertinentes ao tipo de exame solicitado, o que significa que são os médicos burocratas, e não advogados, que estão lidando
com tais situações (pelo menos em primeira e segunda instâncias).
\item E o maior problema de todos. A Cassi é obrigada a utilizar o código de preços da Tuss, e precisa emitir tal código nas notas. 
Esse código não é bem especificado e se assemelha muito ao \textbf{escambo}. Aqui, darei um exemplo, pois explicar esse código 
é impossível uma vez que ele não faz sentido. Imagine que você vai à Guanabara e pede um quibe. O quibe custa cinco reais, porém o código
de barras do quibe não existe no sistema. A Guanabara te cobra cinco reais e fala \textbf{``Quando você for declarar o imposto de renda, não diga
que comprou um quibe. Diga que você comprou três pastas de dente e duas agulhas de crochê"}. O Tuss não especifica preço para o sequenciamento
de BRCA por \textbf{NGS}, apenas para sequenciamento por \textbf{Sanger}, que é um método completamente diferente (exemplo possivelmente irreal). 
O médico burocrata então deve emitir a nota dizendo que o custo será de \textbf{``10000 x Sequenciamento Sanger"}. 
Além dessa nota ser ridícula (fato com que os médicos concordam), ela não reflete a realidade do exame realizado.
\end{itemize}

Além desses problemas, existem outros dois. As normas e especificações usadas pela Cassi estão defasadas. A Tuss foi atualizada, porém a Cassi não
adotou ainda e não leu o que mudou nas novas versões. Se me lembro bem, estão usando a Tuss 5, e hoje temos já a Tuss 7. O outro problema é que
será criada uma nova tabela/entidade chamada Tiss, que visa resolver tais problemas. Como isso impactará a Cassi, não temos idéia. Mas vale lembrar que
tais normas estão, aparentemente, sendo feitas sem que os laboratórios que \textbf{concretizam} as normas sejam consultados, o que pode levar a
normas mal especificadas e desnecessárias (o que já ocorreu com a internet e o modelo OSI, um estudo de caso interessante). E, eventualmente serão
quatro entidades metendo o bedelho em tudo quanto é lugar.

No segundo dia, o Euclides mudou a abordagem de minicurso para debate/mesa redonda/\textit{brainstorm} com os participantes. Isso foi muito construtivo pois pudemos
construir algumas idéias juntamente com os participantes. E ficou claro que, nesse momento, as preocupações não são de caráter financeiro e sim de caráter
burocrático. 
 
Dessa forma, proponho as seguintes perguntas/tópicos que podem ser interessantes. Note que muitos deles foram discutidos lá, mas como você não estava presente,
os coloco aqui. E tomei a lberdade de colocar alguns tópicos que surgiram em minha cabeça no momento de redação da carta.
\begin{itemize}
\item Sistema para gerir a Cassi.
\item Diluir burocracia da Cassi através da participação da DNAConsult/QGene/MD Genetics.
\item Participar na construção de novas normas/códigos em conjunto com ANS e outras entidades.
\item Formulário para padronizar pedidos de exames.
\item O mesmo formulário seria usado para transferir um pouco de responsabilidades da Cassi e dos laboratórios para os médicos.
\item Impacto do formulário benéfico pois o médico deverá \textbf{saber o que está pedindo}.
\item Consultoria.
\item Auditoria (outros laboratórios cobrando preços exorbitantes e prestando serviços não condizentes ou de ma qualidade).
\item Tipos de exames prestados (4800 ou genes específicos).
\item Redução burocrática através do exame 4800.
\item Modelo de negócios sobre os 4800 (preço de sequenciamento e, posteriormente, preço de consultas/análises sem sequenciamento).
\item Logística de coleta de amostras para exames.
\item Meios/portais para que médicos se atualizem no que diz respeito à genética.
\item Pacotes de exames.
\item Sequenciamento pode ser considerado uma \textit{commodity}?
\item Análise do sequenciamento pode ser visto como \textit{commodity}?
\item Independência entre nosso TI e TI da Cassi.
\end{itemize}

\vspace{2\parskip} % Extra whitespace for aesthetics
\closing{Atenciosamente,}
\vspace{2\parskip} % Extra whitespace for aesthetics

\ps{P.S. Fica expresso aqui que o conteúdo da carta é informal e muitas coisas foram explicadas superficialmente e/ou
baseadas em debates verbais. O conteúdo tem o objetivo de fornecer uma visão \textbf{abstrata} sobre os fatos.
} % Postscript text, comment this line to remove it

%\encl{Copyright permission form} % Enclosures with the letter, comment this line to remove it

%----------------------------------------------------------------------------------------

\end{letter}
 
\end{document}
