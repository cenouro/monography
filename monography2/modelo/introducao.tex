\chapter{Introdução}
\label{chap:intro}
\pagenumbering{arabic}  %%% deixe aqui para começar no Nro 1!!!!!!!

% O capítulo de Introdução deve ter no máximo o total de 2 páginas.

\section{Sobre a Empresa}

A Raccoon Marketing Digital\cite{Raccoon} é uma empresa de publicidade online
com foco em performance, atuando no mercado
há pouco mais de dois anos.

\emph{Search engines} utilizam diversos algoritmos e variáveis para
ponderar a qualidade de sites. O valor pago para que anúncios
sejam realizados no Google, por exemplo, depende da qualidade do
site.

\emph{Performance} é uma área do marketing digital que tem por objetivo
aumentar vendas e diminuir gastos pagos por cada anúncio, ao
invés de simplesmente aumentar a verba destinada aos anúncios.
Performance é costumeiramente mensurada pelo conceito de
\gls{roi}.

Como parte da estratégia da empresa, a Raccoon também conta com
um setor de TI responsável pelo desenvolvimento de softwares que
possam automatizar certas tarefas rotineiramente realizadas pelo setor 
de marketing e softwares que possam melhorar o \gls{roi} de clientes.

Atualmente, a empresa conta com aproximadamente quinze
funcionários no setor de TI e quarenta e cinco no setor de
marketing digital.

Não existe plano de carreira e esse é um aspecto da empresa que
necessita de amadurecimento. O valor da bolsa de estágio é
consideravelmente maior do que o valor pago por outras empresas na
região; porém, a ausência de plano de carreira implica que
cargos não possuem responsabilidades bem definidas e a
perspectiva de crescimento profissional (no sentido de promoções,
não de conhecimento adquirido) é extremamente vaga.

% Comente sobre a empresa: o setor de atuação, o porte, a missão e outras informações que você julgue necessárias, como por exemplo plano de carreira e premiações.

% Em toda a monografia, você pode fazer uso de  referencias bibliográficas para apresentar URLs \cite{WAI}, livros \cite{Dahl:1972} \cite{Hopcroft:1969} ou artigos utilizados no decorrer do estágio, publicados em periódicos \cite{Almorsy:2012} ou revistas \cite{MichettiJavaMagazine2013}.

\section{Sobre o Processo Seletivo}

O processo seletivo consistiu em três entrevistas. A primeira
entrevista foi realizada por dois membros do setor de TI. A entrevista
consistiu em perguntas para avaliar meu entendimento sobre
conceitos de programação em geral, banco de dados, expressões
regulares, Linux e familiaridade com \emph{manpages} e sobre experiências anteriores.

As demais entrevistas foram realizadas com os dois fundadores da
empresa, cada uma com um sócio. Foram entrevistas normais, e, em
uma delas, precisei resolver alguns problemas estilo
\textit{guesstimate}.

% Comente sobre como foi o processo seletivo para admissão no estágio. Discuta como você se preparou para ele.

\section{Apresentação da monografia}

Os próximos capítulos irão retratar de forma mais detalhada as atividades realizadas por mim durante o estágio.

No Capítulo~\ref{chap:atividadesPlanejadas}, descrevo brevemente como foi o planejamento das atividades. É apresentado também
como foi meu treinamento.

No Capítulo~\ref{chap:atividadesRealizadas}, apresento as atividades realizadas e ferramentas sendo utilizas para desenvolver tais atividades. 

No último capítulo, são levantadas as conclusões sobre o estágio e a relação com 
o curso de graduação, indicando melhorias e perspectivas para o futuro, tanto para o curso 
quanto para o trabalho.

% Escreva um parágrafo que resume cada um dos demais capítulos da monografia, fazendo referência a cada um deles -- como no exemplo que segue. Indique também a existência de apêndices e anexos, se houver.

% No Capítulo~\ref{chap:atividadesPlanejadas} é primeiramente sumarizado o planejamento das atividades previstas para todo o estágio, com o cronograma correspondente. A seguir, são apresentados  os treinamentos previstos.
