\chapter{Conclusão}
\label{chap:conclusao}

% Este capítulo não deve ultrapassar o total de 2 páginas.

\section{Benefícios para o crescimento profissional}

O estágio me proporcionou a oportunidade de aprimorar áreas de
meu conhecimento e o aprendizado de
conceitos, técnicas, metodologias e linguagens novas. Estou
satisfeito especialmente pela oportunidade de trabalhar com um
banco de dados NoSQL.

Tive também a decepção de trabalhar utilizando a linguagem Python.
Python foi projetada com foco em minimalismo. A
linguagem é amplamente utilizada em trabalhos acadêmicos;
também é utilizada pelo próprio Google na implementação de
scripts, quando cabível. Infelizmente, ainda é comum promover a
linguagem como ``a linguagem que o Google utiliza''. Google
utiliza uma pletora de linguagens, e seus vários projetos open
source podem ser vistos em seu repositório. Em minha opinião,
Python está longe de ser a primeira escolha para o
desenvolvimento de sistemas. Esse é um nicho melhor abordado por
linguagens como Java e Ruby. Tal parágrafo pode ser considerado
irrelevante, já que Python é uma linguagem \emph{Turing
Complete}, mas ainda o considero cabível pois a linguagem teve
sim um impacto desmotivador em mim.

% Comente como o estágio foi importante para o seu crescimento profissional. 

% Se houve problemas e/ou decepções, discuta-os nesta seção.

\section{Considerações sobre o curso de graduação}

O curso me forneceu fortes bases teóricas para o mercado de
trabalho. Ainda assim, eu considero que o curso não foi bem
planejado. Muito tempo é perdido em sala de aula, e pouco tempo é
investido tarefas que realmente agregam conhecimento.

O mercado de trabalho é capaz de agregar conhecimento e
experiência notavelmente mais rapidamente do que as aulas. Isso é
de se esperar; contudo, a universidade deveria incentivar o
estágio desde períodos iniciais da graduação. Ao invés disso, o
que se tem é um foco \emph{exagerado} em matérias de cálculo
(seis créditos, mais do que as matérias de computação, se não
todas) e períodos longos em sala de aula (algumas aulas podendo
ter duração de até quatro horas). Infelizmente, para a maioria
dos alunos, o estágio deixa de ser uma forma de complementar o
próprio conhecimento e passa a ser mais uma matéria obrigatória
que (de acordo com o período ideal do curso) só deveria ser
cursada no último ano.

% Avalie o seu curso de graduação, em particular em termos de preparação para o mercado de trabalho. 

% Discuta também a importância das disciplinas do curso para o estágio realizado.

\section{Sugestões para o curso de graduação}

Ementas de disciplinas deveriam ser revisadas e atualizadas para
condizer com o que se utiliza no mercado de trabalho.
Infelizmente, uma série de disciplinas contém conteúdo
irrelevante para o aluno. Por exemplo, a disciplina de Engenharia
de Software I abrange o Modelo Cascata. Acredito que tal modelo
devesse ser abordado, no máximo, em alguma disciplina específica
de história da computação; a disciplina de Engenharia de Software
poderia abordar, por exemplo, \emph{workflows} adotados em
sistemas de versionamento de código, como o \emph{git}. Tal
conteúdo é muito mais relevante para a vida profissional do aluno
do que modelos obsoletos.

A ênfase em matérias de cálculo também deveria ser reanalisada.
Tais matérias são importantes, mas a carga atual (seis créditos)
é um tanto quanto exagerada. Acredito que realocar uma parcela de
tal carga para matérias como estatística, matemática discreta ou 
uma miscelânea de matérias de computação seria algo benéfico.

O curso também deveria focar mais matérias pertinentes ao
desenvolvimento de sistemas web e APIs. Existem disciplinas que
abrangem tal contudo, mas tais disciplinas são optativas.

Como um todo, os trabalhos deveriam ser repensados. Não vejo
motivo, por exemplo, em um trabalho onde o aluno deva implementar
uma versão reduzida do \emph{Microsoft Paint} utilizando Java e
facilidades de \emph{drag 'n drop} da IDE NetBeans. Não é algo
comum, já que a maioria das disciplinas possuem trabalhos relevantes; mas ainda existe a falta de continuidade. Seria interessante,
por exemplo, que o aluno implementasse um sistema inteiro ao
longo do curso; um pouco em cada disciplina.

Por fim, a universidade deveria levar em conta aspectos
extracurriculares dos alunos. Cursei uma disciplina onde o
professor adicionou meio ponto na média final de alunos que
participaram de uma determinada campanha para doação de sangue.
Foi uma ideia um tanto quanto interessante; de qualquer forma, me
pergunto porque nenhum professor adiciona pontos extras para
alunos que contribuem para projetos open source ou para alunos
ativos no site Stackoverflow\cite{Stack}, por exemplo.

% Apresente sugestões construtivas para o curso de graduação, justificando cada uma.

\section{Planos para o futuro}

Pretendo continuar trabalhando na empresa. Apesar da inexistência
de plano de carreira, a aquisição de conhecimentos está
condizente com o que eu havia planejado para meu crescimento
profissional.

% Comente quais são seus planos depois do término do estágio, a curto e médio prazos.
