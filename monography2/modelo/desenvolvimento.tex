\chapter{Desenvolvimento do Trabalho}
\label{chap:atividadesRealizadas}

% Este capítulo deve ter de 4 a 10 páginas.

\section{Atividades Realizadas}

\subsection{NapPy}

O desenvolvimento inicial do NapPy foi realizado por mim e um
colega de trabalho. Como não tínhamos familiaridade com a API do
Google AdWords e nem com MongoDB, optamos por realizar o
desenvolvimento em \emph{pair programming}. Também seguimos a
convenção da empresa, adotando as metodologias SCRUM e
\gls{tdd}.

\subsection{Refatoração do NapPy}

Eventualmente, meu colega decidiu sair do estágio por motivos
pessoais. Assim sendo, solicitei que um sprint fosse alocado para
que eu pudesse realizar benchmarks e me familiarizar com partes
da code base que foram implementadas principalmente por meu
colega pois eu continuaria como único desenvolvedor do software.

Os resultados do benchmark deixaram muito a desejar. NapPy, basicamente, 
realizava três requisições por campanha para os servidores do Google. Tais
requisições eram o gargalo do sistema e, para enviar múltiplas
campanhas, NapPy realizaria um número enorme de requisições.

Com tais resultados, solicitei que mais dois sprints fossem
alocados para que o código fosse refatorado. A refatoração que
planejei tinha por objetivo melhorar o desempenho do sistema e
utilizar o serviço de labels da API do Google para simplificar a
lógica de negócios do software.

\subsubsection{Otimização}

Como mencionado anteriormente, o gargalo do software era o
número de requisições enviadas para o Google. NapPy realizava
três requisições por campanha. A primeira requisição lidava com
o serviço de campanhas da API; a segunda, com o serviço de grupos
de anúncios; e a terceira, com o serviço de anúncios.

Tais requisições precisavam ser realizadas por campanha pois esta
era a forma como o banco de dados foi modelado. A modelagem foi
feita visando utilizar \emph{embedded documents}, uma das formas
de armazenar dados em MongoDB.

Para otimizar o desempenho, o banco foi remodelado. A nova
modelagem (ainda em MongoDB) contém três tabelas. Uma tabela
armazena campanhas; a segunda, armazena grupos de anúncios e o
\emph{id} da campanha à qual o grupo de anúncio pertence; a
terceira, armazena anúncios, o \emph{id} da campanha e o
\emph{id} do grupo de anúncio aos quais o anúncio pertence.
Essas três tabelas são muito mais próximas da estrutura de árvore
utilizada pelo Google para representar campanhas e anúncios do
que a modelagem que utilizava documentos aninhados.

Com a nova modelagem, ficou trivial otimizar o software.
Primeiro, uma requisição era enviada com uma lista de
campanhas. Depois, uma segunda requisição contendo todos os
grupos de anúncio pertencentes às campanhas enviadas
anteriormente. Por fim, uma terceira requisição enviava a lista
de anúncios pertinentes às entidades enviadas anteriormente. Três
requisições seriam capazes de realizar toda a lógica de negócio.
Na prática, porém, o número de requisições pode ser maior. A API
do Google impõem certas limitações. Dessa forma, NapPy realiza
requisições com, no máximo, 5000 operações.

Mesmo com a simplicidade da nova lógica de percorrer a estrutura
de campanhas, a code base do sistema já tinha um tamanho
considerável. Para evitar problemas durante a refatoração, segui
técnicas descritas em \cite{Legacy}. Dessa forma, o sistema
continuava funcionando, e a refatoração poderia ser comparada com
a antiga implementação para validação de resultados.

Após a refatoração, executei os benchmarks novamente. A versão
anterior do NapPy demorava 2 minutos para criar uma campanha. A
nova versão é capaz de criar 20000 campanhas em 20 minutos.

A refatoração é considerada um sucesso. Não somente melhorou o
desempenho do software como também simplificou em muito a rotina
 que percorre a estrutura de árvore das campanhas. Isso, por
sua vez, permitiu que sub-rotinas de tratamento de erros fossem
implementadas de forma concisa.

\subsubsection{Utilização do serviço de labels}

A API do Google Adwords contém um serviço específico para
lidar com labels. Labels podem ser adicionadas ou removidas às
entidades nos servidores do Google e
utilizadas para a realização de queries em outras tabelas.

Após a execução do NapPy, toda a estrutura de campanhas e
anúncios estaria concretizada nos servidores do Google, e os
usuários poderiam proceder normalmente com as otimizações de
campanhas e anúncios. Porém, os usuários utilizam uma ferramenta
do Google para isso. Dessa forma, NapPy teria a responsabilidade
extra de manter a coerência entre os dados no Google e dados em
MongoDB.

Tal lógica de coerência seria extremamente complexa para
implementar. Para lidar com tal problema, cogitava-se a
implementação de uma interface gráfica para ser utilizada ao
invés da ferramenta do Google.

Conjuntamente com a remodelagem do banco de dados, propus que 
labels fossem utilizadas para lidar com a coerência de dados. A
ideia seria que o usuário continuasse utilizando a ferramenta do
Google, mas que adicionasse labels pré-definidas para certas
operações quando cabível. Por exemplo, se o usuário desejasse que
um anúncio parasse de ser atualizado através das rotinas do
NapPy, bastaria adicionar uma label específica ao anúncio. Labels não
resolveriam completamente o problema, mas simplificariam em muito
o código necessário para cobrir todos os casos de uso.
Essa feature não foi implementada até o presente momento.

% Descreva quais atividades foram realizadas durante o projeto. 

% Para cada atividade realizada, discuta também qual foi a dinâmica de trabalho --- por exemplo, se foi utilizada a metodologia SCRUM ou algo similar.

% Se apropriado, você apresentar algoritmos ou códigos, como o ilustrado na Listagem~\ref{src:code1}, mas deve explicar o seu funcionamento no texto.

% \begin{lstlisting}[
%   language=C,
%   caption=Exemplo de código,
%   label=src:code1
% ]
% #include <stdio.h>
% int main(){ 
%     int i=0, j=1;
%     printf("i:%d j:%d\n",i,j);
%     return;
% }
% \end{lstlisting}

% Você também pode fazer uso de figuras (como a Figura~\ref{fig:fig1}), mas deve explicar a figura no texto.

% \begin{figure}[!ht]
%   \rule[1ex]{\textwidth}{0.25pt}
%   \centering\includegraphics[width=0.25\textwidth]{img/logoICMC.png}
%   \caption [Exemplo de figura]
%   {Exemplo de figura}\label{fig:fig1}
%   \rule[1ex]{\textwidth}{0.25pt}
% \end{figure}


\section{Problemas resolvidos}

Ao término do estágio, o sistema NapPy se encontrava em estágio
avançado de desenvolvimento e os requisitos de tempo de execução
foram atingidos.

% Descreva quais problemas foram resolvidos. 


\section{Técnicas, métodos e tecnologias envolvidas}

\subsection{MongoDB}

MongoDB\cite{Mongo} é um banco de dados \emph{NoSQL}, baseada em \emph{documentos}. 
A definição de tais termos foge do escopo desse trabalho, e pode
ser encontrada em \cite{NoSQL}.

MongoDB utiliza objetos JSON para representar documentos, que podem ser
facilmente manipulados em Python. Queries também são descritas
como objetos JSON.

MongoDB foi escolhido por sua simples utilização e pela
possibilidade de ser escalado horizontalmente.

\subsection{Python}

NapPy foi implementado na linguagem Python, versão 3.

\subsubsection{PyMongo}

PyMongo\cite{Pymongo} é o driver nativo de MongoDB para Python. A biblioteca é
simples de ser utilizada. Queries em Mongo são objetos JSON, e
dicionários em Python podem facilmente ser convertidos para
objetos JSON. O mesmo vale para documentos resultantes de
queries.

\subsubsection{MongoEngine}

MongoEngine\cite{Mongoengine} é uma ORM para MongoDB e Python. MongoEngine foi
projetada para ser similar a outras ORMs 
utilizadas em Django, um framework em Python para sistemas
\gls{mvc}.

MongoEngine é um tanto quanto mais abstrata do que PyMongo, até
mesmo em como descrever queries. A biblioteca foi utilizada para
implementar os modelos (inclusive os índices) utilizados pelo NapPy;
mas PyMongo foi utilizada para implementar queries já que a
abstração extra de MongoEngine seria desnecessária (podendo até
mesmo tornar o código menos legível).

\subsubsection{googleads-python-lib}

googleads-python-lib\cite{Googleads} é a biblioteca cliente para Python. O
repositório da biblioteca contem a implementação de diversos
exemplos, inclusive como gerar credenciais de OAUTH 2.0 para se
acessar a API do Google.

\subsection{Google AdWords API}

A API do Google AdWords\cite{AdWords} permite que dados do serviço de AdWords
sejam acessados de forma programática. A API é composta por
diferentes serviços, e.g. AdGroupAdService é o serviço utilizado
para lidar com anúncios. 

A API também disponibiliza a \gls{awql}\cite{AWQL}, 
para queries complexas. Queries simples não precisam
utilizar AWQL; bibliotecas clientes possuem classes que
implementam o \emph{Query Object design pattern}.

A API impõem certas limitações em sua utilização. Existem limites
nos números de requisições por minuto, no número de requisições
realizadas concorrentemente e no número de operações por
requisição. \gls{awql} não implementa JOIN e nem GROUP BY.

\subsection{Test Driven Development}

O desenvolvimento foi realizado utilizando-se
\gls{tdd}. Tal prática consiste em escrever
testes para o sistema antes que o código em si seja escrito.
Dependendo de como os testes são escritos, eles podem até mesmo
servir como documentação do sistema.

A fase inicial do projeto contou com uma pletora de testes de
unidade. Tais testes visam especificar o comportamento de
componentes do sistema de forma isolada. Como testes de unidade
podem ter grande impacto na
produtividade\cite{Coplien}, optei por deixá-los
sob versionamento por referência, dando apenas continuidade na
escrita de testes de integração.

% Descreva os métodos, técnicas  e tecnologias
% envolvidos  ou que  foram utilizados  para a  condução das  atividades
% durante  o estágio.  Por exemplo:  (i)  no caso  dos métodos,  pode-se
% apresentar XP (eXtreme Programming) e SCRUM, métodos empregados para o
% teste de  sistemas, entre  outros; (ii) No  caso de  técnicas, pode-se
% descrever técnicas da UML, técnicas de teste, entre outros; e (iii) no
% caso  de  tecnologias,  pode-se descrever  ferramentas  (por  exemplo,
% Spring, Hibernate,  Struts, entre  outros), linguagens  de programação
% (por exemplo, Java), padrões de projeto utilizadas, entre outros.

% Faça  referências bibliográficas  atuais e  de fontes
% relevantes  (evite  sites,  privilegie   livros  ou  artigos);  mostre
% diferentes  tecnologias e  faça comparações,  quando for  o caso.  Não
% esquecer de citar a fonte corretamente, por exemplo~\cite{MichettiJavaMagazine2013}.

\section{Impacto}

Eventualmente, o desenvolvimento do NapPy foi cancelado. O
projeto não possuía escopo bem definido e outros projetos com
maior prioridade surgiram. Ainda assim, a code base ainda é
utilizada como referência por ser escrita de forma bem
estruturada e conter trechos de código que podem facilmente serem
reutilizados (especialmente rotinas de benchmark).

O desenvolvimento do sistema também serviu para evidenciar certas
falhas no processo de desenvolvimento de softwares da empresa
como um todo.

% Quais foram as pessoas/entidades afetadas por esses resultados? Quem são  as pessoas/entidades que potencialmente serão afetadas?  

% Fale sobre a relevância da solução do(s) problema(s) para a empresa e seu(s) cliente(s).

% Você pode omitir informações sigilosas.

% Discuta também sobre a relevância para sua formação a  participação na solução do(s) problema(s).

\section{Problemas não resolvidos}

A feature que utiliza labels não foi implementada. O código que
lidava com palavras-chave também não foi refatorado e incluído na
versão refatorada pois o desenvolvimento já havia sido cancelado. 
